% !TeX encoding = UTF-8

%% ------------------------------------------------------------------------
%% Copyright (C) 2021-2023 SJTUG
%% 
%% SJTUBeamer Example Document by SJTUG
%% 
%% SJTUBeamer Example Document is licensed under a
%% Creative Commons Attribution-NonCommercial-ShareAlike 4.0 International License.
%% 
%% You should have received a copy of the license along with this
%% work. If not, see <http://creativecommons.org/licenses/by-nc-sa/4.0/>.
%%
%% For a quick start, check out src/doc/sjtubeamerquickstart.tex
%% Join discussions: https://github.com/sjtug/SJTUBeamer/discussions
%% -----------------------------------------------------------------------

\documentclass[xcolor=table,dvipsnames,svgnames,aspectratio=169]{ctexbeamer}
% 可以通过 fontset=macnew / fontset=ubuntu / fontset=windows 选项切换字体集;
% 如遇无法显示的数学符号,尝试对 ctexbeamer 文档类添加 no-math 选项;
% 写纯英文幻灯片可以改用 beamer 文档类。

\usepackage{tikz}
\usepackage[normalem]{ulem}
\usetikzlibrary{arrows}
\usepackage{amsmath}
\usepackage{graphicx}
\usepackage{hologo}
\usepackage{colortbl}
\usepackage{shapepar}
\usepackage{hyperxmp}
\usepackage{booktabs}
\usepackage{listings}
\usepackage{tipa}
\usepackage{multicol}
\usepackage{datetime2}
\usepackage{fontawesome5}
\usepackage{hyperref}
% 量和单位
\usepackage{siunitx}
%si格式
\sisetup{
  separate-uncertainty = true,
  inter-unit-product = \ensuremath{{}\cdot{}}
}

% 参考文献设置,使用 style=gb7714-2015 样式为标准顺序编码制,
% 使用 style=gb7714-2015ay 样式可以改为著者-出版年制。
% \usepackage[backend=biber,style=gb7714-2015]{biblatex}
% \addbibresource{ref.bib}

% 该行指定了图像的额外搜索路径
\graphicspath{{figures/}}

\hypersetup{
  pdfcopyright       = {Licensed under CC-BY-SA 4.0. Some rights reserved.},
  pdflicenseurl      = {http://creativecommons.org/licenses/by-sa/4.0/},
  unicode            = true,
  psdextra           = true,
  pdfdisplaydoctitle = true
}

\pdfstringdefDisableCommands{
  \let\\\relax
  \let\quad\relax
  \let\hspace\@gobble
}

% \renewcommand{\TeX}{\hologo{TeX}}
% \renewcommand{\LaTeX}{\hologo{LaTeX}}
% \newcommand{\BibTeX}{\hologo{BibTeX}}
% \newcommand{\XeTeX}{\hologo{XeTeX}}
% \newcommand{\pdfTeX}{\hologo{pdfTeX}}
% \newcommand{\LuaTeX}{\hologo{LuaTeX}}
% \newcommand{\MiKTeX}{\hologo{MiKTeX}}
% \newcommand{\MacTeX}{Mac\hologo{TeX}}
% \newcommand{\beamer}{\textsc{beamer}}
% \newcommand{\XeLaTeX}{\hologo{Xe}\kern-.13em\LaTeX{}}
% \newcommand{\pdfLaTeX}{pdf\LaTeX{}}
% \newcommand{\LuaLaTeX}{Lua\LaTeX{}}
% \def\TeXLive{\TeX{} Live}
% \let\TL=\TeXLive

% \newcommand{\SJTUThesis}{\textsc{SJTUThesis}}
% \newcommand{\SJTUThesisVersion}{2.0.3}
% \newcommand{\SJTUThesisDate}{2023/9/25}
% \newcommand{\SJTUBeamer}{\textsc{SJTUBeamer}}
% \newcommand{\SJTUBeamerVersion}{3.0.0}
% \newcommand{\SJTUBeamerDate}{2022/11/22}

% \newcommand\link[1]{\href{#1}{\faLink}}
% \newcommand\pkg[1]{\texttt{#1}}

% \def\cmd#1{\texttt{\color{structure}\footnotesize $\backslash$#1}}
% \def\env#1{\texttt{\color{structure}\footnotesize #1}}
% \def\cmdxmp#1#2#3{\small{\texttt{\color{structure}$\backslash$#1}\{#2\}
% \hspace{1em}\\ $\Rightarrow$\hspace{1em} {#3}\par\vskip1em}}

\usetheme[maxplus,blue,light]{sjtubeamer}
\setbeameroption{show notes on second screen=bottom}
% 使用 maxplus/max/min 切换标题页样式
% 使用 red/blue 切换主色调
% 使用 light/dark 切换亮/暗色模式
% 使用外样式关键词以获得不同的边栏样式
%   miniframes infolines  sidebar
%   default    smoothbars split	 
%   shadow     tree       smoothtree
% 使用 topright/bottomright 切换徽标位置
% 使用逗号分隔列表以同时使用多种选项

% \setbeamertemplate{background}{}
% 对于 max 主题,如果需要关闭正文背景图,请取消注释上一行。

% \tikzexternalize[prefix=build/]
% 如果您需要缓存 tikz 图像,请取消注释上一行,并在编译选项中添加 -shell-escape。

\lstset{
  language=[LaTeX]TeX,           % 更改高亮语言
  texcsstyle=*\color{cprimary},  % 只在高亮 LaTeX 语言时必须
  tabsize=2,
  basicstyle=\ttfamily\scriptsize,%
  keywordstyle=\color{cprimary},%
  stringstyle=\color{csecondary},%
  commentstyle=\color{ctertiary!50!gray},%
  breaklines,%
}
\logo{}
\author{熊家辉}
\institute[萨塞克斯人工智能学院]{浙江工商大学}
% \date{\the\year 年 \the\month 月}
\date{2024 年 6 月 5 日}
% \date{\today}
\subject{RISCOF 初步}
\keywords{CI, secure}

\title[Tabby Terminals] % 页脚显示标题
{\textbf{Tabby Terminals 系列工具}} % 首页标题

\subtitle{PLCT Lab 每周技术分享}

\begin{document}

% 使用节目录
\AtBeginSection[]{
  \begin{frame}
    %% 使用传统节目录,也可以将 subsectionstyle=... 换成 hideallsubsections 以隐藏所有小节信息
    % \tableofcontents[currentsection,subsectionstyle=show/show/hide]
    %% 或者使用节页
    \sectionpage
  \end{frame}
}

% 使用小节目录
\AtBeginSubsection[]{		       % 在每小节开始
  \begin{frame}
    %% 使用传统小节目录
    % \tableofcontents[currentsection,subsectionstyle=show/shaded/hide]
    %% 或者使用小节页
    \subsectionpage
  \end{frame}
}

\maketitle

% \begin{frame}
%   \frametitle{来源}
%   \begin{thebibliography}{00}
%     \setbeamertemplate{bibliography item}[online]
%     \bibitem{} Alexara Wu.
%     \newblock 如何使用 \LaTeX{} 排版论文[EB/OL].
%     \newblock 2021.
%     \url{https://github.com/sjtug/sjtulib-latex-talk/tree/alexara-2021}
%   \end{thebibliography}

%   \vspace*{2ex}

%   \begin{itemize}
%     \item 本示例文档的源码结构适用于简短的单次报告,仅展示 \beamer{} 文档类的通
%           用功能,更多地在使用 \SJTUBeamer{} 的样式信息。
%     \item 为发挥 \SJTUBeamer{} 的全部功能,参见发布区
%           \link{https://github.com/sjtug/SJTUBeamer/releases} 的快速入门、用户手
%           册与开发文档。
%     \item 就制作一组讲座而言,相关源码结构可以参考新讲座
%           \link{https://github.com/sjtug/sjtulib-latex-talk/tree/logcreative-2022}。
%           新讲座使用了社区版主题的同时也展示了 \SJTUBeamer{} 的特殊用法。
%   \end{itemize}

% \end{frame}

\begin{frame}{目录}
  \tableofcontents[hideallsubsections]	% 隐藏所有小节信息
\end{frame}

\section{Tabby Terminal}

\subsection{简介}

\begin{frame}[allowframebreaks]
  \frametitle{Tabby 简介}

  \href{https://github.com/Eugeny/tabby}{Tabby} (前身是 Terminus) 是一个可高度配置的终端模拟器和 SSH 或串口客户端,支持 Windows,macOS 和 Linux。

  \begin{itemize}
    \item 集成 SSH,Telnet 客户端和连接管理器
    \item 集成串行终端
    \item 定制主题和配色方案
    \item 完全可配置的快捷键和多键快捷键
    \item 分体式窗格
    \item 自动保存标签页
    \item 支持 PowerShell(和 PS Core)、WSL、Git-Bash、Cygwin、MSYS2、Cmder 和 CMD
    \item 在 SSH 会话中通过 Zmodem 进行直接文件传输
    \item 完整的 Unicode 支持,包括双角字符
    \item 不会因快速的输出而卡住
    \item Windows 上舒适的 shell 体验,包括 tab 自动补全(通过 Clink)
    \item 为 SSH secrets 和设置集成了加密容器
    \item SSH、SFTP 和 Telnet 客户端可用作 Web 应用(也可托管)
    \item 如果在 Tabby.exe 所在的目录创建一个名为data文件夹,Tabby 将可以在 Windows 上作为便携式的应用程序运行。
  \end{itemize}

\end{frame}

\begin{frame}
  \frametitle{Tabby 是什么}

Tabby 是 Windows 标准终端 (conhost)、PowerShell ISE、PuTTY、macOS Terminal.app 和 iTerm 的替代品

Tabby 不是一个全新的 shell,也不是 MinGW 或 Cygwin 的替代品。它也不是轻量级的。 在 Windows 上运行一个 Tabby 大约需要 $\SI{360}{MB}$ 内存。
\end{frame}

\subsection{功能}

\begin{frame}
  \frametitle{Tabby 支持的 SSH}

  \begin{itemize}
    \item 带有连接管理器的 SSH2 客户端
    \item X11和端口转发
    \item 自动跳转主机管理
    \item 代理转发(包括 Pageant 和 Windows 原生 OpenSSH 代理)
    \item 登录脚本
  \end{itemize}

\end{frame}

\begin{frame}
  \frametitle{Tabby 支持的串行终端}

  \begin{itemize}
    \item 保存连接
    \item 逐行读取的输入支持
    \item 可选的十六进制逐字节输入和十六进制转储输出
    \item 换行转换
    \item 自动重连
  \end{itemize}

\end{frame}
\begin{frame}[allowframebreaks]
  \frametitle{Tabby 的一些插件}
  插件和主题可以直接在 Tabby 设置中安装。

  \begin{itemize}
    \item clickable-links - 使终端中的路径和 URL 可点击
    \item docker - 连接 Docker 容器
    \item title-control - 允许通过提供要删除的前缀、后缀和/或字符串来修改标签页的标题
    \item quick-cmds - 快速向一个或所有标签页发送命令
    \item save-output - 将终端输出记录到文件中
    \item sync-config - 将配置同步到 Gist 或 Gitee
    \item clippy - 一个可以一直烦你的示例插件
    \item workspace-manager - 允许根据给定的配置创建自定义工作区配置文件
    \item search-in-browser - 从 Tabby 标签页带有选中的文本来打开系统默认浏览器
    \item sftp-tab - 为ssh连接打开类似SecureCRT的sftp标签页
  \end{itemize}

\end{frame}

\section{Tabby Web}

\subsection{简介}

\begin{frame}
  \frametitle{Tabby-web 简介}

  \href{https://github.com/Eugeny/tabby-web}{Tabby-web} 将 Tabby Terminal 作为 Web 应用程序提供,同时管理多个配置文件、身份验证并通过单独的网关服务提供 TCP 连接。

\end{frame}

\subsection{需求}

\begin{frame}
  \frametitle{Tabby 是什么}
  
  \begin{itemize}
    \item Python 3.7+
    \item 带 Django 支持的数据库(MariaDB, Postgres, SQLite)
    \item 存储(如本地存储,S3存储等,只要 \lstinline|fsspec| 支持即可)
  \end{itemize}
\end{frame}

\subsection{使用 docker 启动}

\begin{frame}[fragile]
  \frametitle{Tabby 支持的 SSH}
  \begin{codeblock}[language=bash]{工具链编译}
docker-compose up -e SOCIAL_AUTH_GITHUB_KEY=xxx -e SOCIAL_AUTH_GITHUB_SECRET=yyy
  \end{codeblock}

  \note{
    OAuth credentials from GitHub, GitLab, Google or Microsoft for authentication.
    For SSH and Telnet: a tabby-connection-gateway to forward traffic.
    Docker BuildKit: export DOCKERBUILDKIT=1

    will start Tabby Web on port 9090 with MariaDB as a storage backend.

  For SSH and Telnet, once logged in, enter your connection gateway address and auth token in the settings.}
  
\end{frame}

\makebottom

\end{document}